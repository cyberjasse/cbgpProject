\documentclass[a4paper, 12pt]{report}
\usepackage[utf8]{inputenc}
\usepackage[left=2cm,right=2cm,top=2cm,bottom=2cm]{geometry}
\usepackage[pdftex]{graphicx}
\usepackage[final]{pdfpages}

\begin{document}
\begin{titlepage}
\begin{center}
\includegraphics[scale=1.50]{UMONS.jpg}\\[0.4cm]
\includegraphics[scale=0.30]{FS_Logo.jpg}\\[3cm]
{\huge Réseayx II - TP3}\\
\vspace{0.9mm}
\rule{9.5cm}{0.5mm}\\[0.5cm]
{\huge \bfseries Tiny Internet Project}\\[0.2cm]
\rule{9.5cm}{0.5mm}\\[7cm]
      \begin{flushleft} \large
        Jason \textsc{Bury} 130538\\
        Alicia \textsc{Galina} 130852\\
        Master 1 en Sciences informatiques\\
      \end{flushleft}
   \vfill
  {\large 11/11/2016}
\end{center}
\end{titlepage}

\section*{Introduction}
Le but du projet est de simuler la topologie de la figure ? avec l'outil cbgp et ensuite de configurer chaque routeur comme ou routeur BGP et de créer des filtres pour chaque routeur.
Pour chaque filtre, on réfléchit pour servir uniquement les intérêts de l'AS en question.

\section{Filtres}
Un client paie à son fournisseur la bande passante utilisée sur le(s) liens qui les connectent, peut importe le sens du trafic.
De plus, certaines routes sont à préférer à d'autres pour des questions de latences (affiché en ms sur chaque lien dans la figure ?).
\subsection{Règles générales}
Voici quelques raisonnements utilisés par plusieurs AS.
\begin{enumerate}
 \item Si on a plusieurs fournisseurs, on fait en sorte que les routes annoncées par un fournisseurs ne soient pas anoncé à un autre.
 Sinon le fournisseur qui reçoit la route fera passer du trafic par chez nous pour le trafic passant par l'autre fournisseur.
\end{enumerate}

\subsection{UCLA}
\end{document}